% ----------------- Documento ----------------- %
% Se define el tipo de documento (en este caso un artículo), en hoja A4 con tamaño de fuente de 11pt, escrito en castellano, e indicando que el documento tendrá páginas distintas a izquierda y derecha ("twoside"):
\documentclass[a4paper, 11pt, spanish, twoside]{article}
% Los demás tipos de documentos así como sus características y opciones pueden consultarse en: https://en.wikibooks.org/wiki/LaTeX/Document_Structure#Document_classes
% ---------------------------------------------- %


%%%%%%%%%%%%%%%%% - PREÁMBULO - %%%%%%%%%%%%%%%%% 

% ------------------ Página -------------------- %
% Se define el tamaño de las páginas, indicando el tamaño de los márgenes superior e inferior ("top" y "bottom"), e izquierdo y derecho ("left" y "right"):
\usepackage[top=2.5cm,bottom=2.5cm,left=2.5cm,right=2.5cm]{geometry}
% Se inserta el comando \raggedbottom para evitar que LaTeX rellene con espacios en blanco aquellas páginas que no alberguen suficiente contenido como para rellenarlas de forma "natural":
\raggedbottom 
% ---------------------------------------------- %


% ------------- Paquetes generales ------------- %
% Se importan distintos paquetes de propósito general:
\usepackage[utf8]{inputenc}
\usepackage[spanish]{babel}
\usepackage{float}
\usepackage{caption}
% ---------------------------------------------- %


% ------------ Paquetes específicos ------------ %
% Se importan distintos paquetes que será utilizados en momentos concretos del documento: 
\usepackage{pdfpages} % Para insertar la portada en formato PDF.
\usepackage{amssymb} % Para símbolos matemáticos.
\usepackage{bm} % Para negrita en símbolos matemáticos.
\usepackage{amsmath} % Para el entorno "split".
\usepackage[hidelinks]{hyperref} % Para urls.
\usepackage{longtable} % Para tablas largas.
\usepackage{graphicx} % Para insertar imágenes.
\usepackage{wrapfig} % Para posicionar imágenes alrededor del texto.
\usepackage{fontawesome5} % Para utilizar iconos de "fontawesome".
\usepackage{pdflscape}  % Para colocar páginas en formato apaisado.
\usepackage[T1]{fontenc}
\usepackage{textcomp}
\usepackage{lmodern} % Soluciona el problema de la mala resolución
% ---------------------------------------------- %


% ---------------- Numeración ------------------ %
\counterwithin{table}{section} % Se numeran las tablas con respecto al capítulo en el que se encuentran.
\counterwithin{figure}{section} % Se numeran las figuras con respecto al capítulo en el que se encuentran.
\counterwithin{equation}{section} % Se numeran las ecuaciones con respecto al capítulo en el que se encuentran.
% ---------------------------------------------- %


% ------------- Página en blanco ----------------%
% Se define un comando (\blankpage) para insertar una página totalmente en blanco (sin número de página, encabezado y pie de página):
\usepackage{afterpage}
\newcommand\blankpage{%
    \null
    \thispagestyle{empty}%
    \newpage}
% ---------------------------------------------- %    


% ----------- Formato de los párrafos -----------%
% Se define el formato de los párrafos:
\setlength{\parindent}{0pt} % Se elimina la sangría en comienzo de párrafo (0pt).
\setlength{\parskip}{1em} % Se define el espacio entre dos párrafos (1em).
% ---------------------------------------------- %    

% -------------- Título adicional -------------- %
% Se añade una profundidad adicional a los títulos (profundidad 4):
\usepackage{titlesec}
\setcounter{secnumdepth}{4} % Se fija en 4 la profundidad de numeración de títulos.
\setcounter{tocdepth}{4} % Se fija en 4 la profundidad de títulos incluidos en el índice.
% Se modifica el formato de \paragraph (título de profundidad 4) para adaptarlo al formato del resto de títulos:
\titleformat{\paragraph}
{\normalfont\normalsize\bfseries}{\theparagraph}{1em}{}
\titlespacing*{\paragraph}
{0pt}{3.25ex plus 1ex minus .2ex}{1.5ex plus .2ex} 
% ---------------------------------------------- %    


% --------- Encabezado y pie de página -------- %
% El encabezado y pie de página forman parte del paquete fancyhdr:
\usepackage{fancyhdr}
\fancyhf{}
\pagestyle{fancy}

% Para solucionar error "headheight is too small":%
\setlength{\headheight}{14.6pt}
\addtolength{\topmargin}{-0.6pt}

% Se ajusta el tamaño de fuente para el encabezado y pie de página (9pt)
\fancyhf{\fontsize{2}{14}\selectfont}

% Contenido del encabezado (\fancyhead):
\fancyhead[RO]{Simulación de operación de una central nuclear} % Texto que se coloca en el encabezado de las páginas impares (O -> 'Odd', o impar) a la izquierda (R -> 'Odd')
\fancyhead[LE]{\nouppercase{\rightmark}} % Texto que se coloca en el encabezado de las páginas pares (E -> 'Even', o par) a la izquierda (L -> 'Left'). \rightmark se utiliza para insertar automáticamente el título de la sección correspondiente, y \nouppercase para que no aparezca todo en mayúsculas (formato por defecto de \rightmark).

% Contenido del pie de página (\fancyfoot):
\fancyfoot[RE]{Escuela  Técnica  Superior  de  Ingenieros  Industriales  (UPM)} % Texto que se coloca en el pie de página de las páginas pares (E -> 'Even', o par) a la derecha (R -> 'Right')
\fancyfoot[LO]{Antonio Dies Beneytez} % Texto que se coloca en el pie de página de las páginas impares (O -> 'Odd', o impar) a la izquierda (L -> 'Left')
\fancyfoot[LE,RO]{\thepage} % El número de página (\thepage) se coloca a la izquierda en las páginas pares y a la derecha en las impares.

% Se indica que sólo se quiere incorporar en \rightmark (utilizado más arriba) el título de la sección (y no de las subsecciones, subsubsecciones, etc.):
\renewcommand{\sectionmark}[1]{\markright{\thesection. #1}}
\renewcommand{\subsectionmark}[1]{}

% Formato de la línea de separación horizontal:
\renewcommand{\headrulewidth}{0.5pt} % Ancho de la línea del encabezado.
\renewcommand{\footrulewidth}{0.5pt} % Ancho de la línea del pie de página.
% ---------------------------------------------- % 


% ----------- Fragmentos de código ------------- %
% El paquete utilizado para insertar fragmentos de código en el documento es listings. En el presente bloque del preámbulo se definen ciertos parámetros de listings con el objetivo de adaptar dicho paquete a código escrito en Python.

\usepackage{listings} % Paquete para insertar código. 
\usepackage{xcolor} % Paquete para definir colores.

% Se definen los distintos colores que se utilizan para resaltar ciertos elementos del código:
\definecolor{codegreen}{rgb}{0.04314,0.6745,0.07843} % Verde.
\definecolor{codegray}{rgb}{0.5,0.5,0.5} % Gris.
\definecolor{codered}{rgb}{0.5373,0.02745,0.06275} % Rojo.
\definecolor{codeblue}{rgb}{0.071,0.0258,0.9882} % Azul.
\definecolor{codepurple}{rgb}{0.6,0.02745,0.5961} % Morado.

% Se define el color de fondo:
\definecolor{backcolour}{rgb}{0.95,0.95,0.92} % Gris oscuro.

% Se define el valor de ciertos parámetros de listings para adaptar dicho paquete a código escrito en Python:
\lstdefinestyle{mystyle}{
    % - General:
    language=Python, % Lenguaje de programación.
    basicstyle=\ttfamily\footnotesize, % Tipografía y tamaño de fuente.
    % - Colores de los distintos elementos del código:
    backgroundcolor=\color{backcolour}, % Color de fondo.  
    commentstyle=\color{codegray}, % Color de los comentarios.
    keywordstyle=\color{codeblue}, % Color de las palabras clave por defecto.
    stringstyle=\color{codegreen}, % Color de los "string"
    % - Palabras clave:
    deletekeywords={print}, % Se elimina "print" del conjunto de palabras clave para posteriormente asignarle el color morado.
    keywordstyle={[2]\ttfamily\color{codeblue}},
    keywords=[2]{as}, % Se añaden las palabras clave de color azul.
    keywordstyle={[3]\ttfamily\color{codepurple}},
    keywords=[3]{True, False, ttk, list, None, dict, zip, range, len, print, float, sum}, % Se añaden las palabras clave de color morado.
    keywordstyle={[4]\bfseries\ttfamily},
    keywords=[4]{_read_excel}, % Se añaden las palabras clave en negrita.
    emph={MyClass,__init__}, % Se añaden las palabras clave enfatizadas.   
    % - Números de línea:
    numberstyle=\tiny\color{codegray}, % Tamaño de fuente y color de los números de línea.
    numbers=left, % Se colocan los números de línea en el lado izquierdo.                 
    numbersep=5pt, % Separación horizontal de los números de línea.
    % - Saltos a la línea, espacios, indentación:
    breaklines=true, % Permitir saltos a la línea. 
    breakatwhitespace=true, % Saltar a la línea únicamente al encontrar espacios.
    postbreak = \mbox{{$\hookrightarrow$}\space}, % Se añade una flecha al cambiar de línea.
    showspaces=false, % No mostrar los espacios. 
    showstringspaces=false, % No mostrar los espacios en los "string".
    keepspaces=true, % Mantener los espacios presentes en el código. 
    tabsize=2, % Tamaño de indentación.
    % - Título:
    captionpos=b % Posición del título del fragmento de código (b=bottom - abajo).
} 
\lstset{style=mystyle} % Se asocia el estilo de listings al estilo que acaba de definirse ("mystyle")

% Se realizan una serie de operaciones complementarias con el paquete listings (su comprensión no es necesaria para manejar dicho paquete):
\makeatletter
\def\lst@OpLiteratekey#1\@nil@{\let\lst@ifxopliterate\lst@if
                             \def\lst@opliterate{#1}}
\lst@Key{opliterate}{}{\@ifstar{\lst@true \lst@OpLiteratekey}
                             {\lst@false\lst@OpLiteratekey}#1\@nil@}
\lst@AddToHook{SelectCharTable}
    {\ifx\lst@opliterate\@empty\else
         \expandafter\lst@OpLiterate\lst@opliterate{}\relax\z@
     \fi}
\def\lst@OpLiterate#1#2#3{%
    \ifx\relax#2\@empty\else
        \lst@CArgX #1\relax\lst@CDef
            {}
            {\let\lst@next\@empty
             \lst@ifxopliterate
                \lst@ifmode \let\lst@next\lst@CArgEmpty \fi
             \fi
             \ifx\lst@next\@empty
                 \ifx\lst@OutputBox\@gobble\else
                   \lst@XPrintToken \let\lst@scanmode\lst@scan@m
                   \lst@token{#2}\lst@length#3\relax
                   \lst@XPrintToken
                 \fi
                 \let\lst@next\lst@CArgEmptyGobble
             \fi
             \lst@next}%
            \@empty
        \expandafter\lst@OpLiterate
    \fi}

\lstset{ 
    literate={á}{{\'a}}1 {é}{{\'e}}1 {í}{{\'i}}1 {ó}{{\'o}}1 {ú}{{\'u}}1
  {Á}{{\'A}}1 {É}{{\'E}}1 {Í}{{\'I}}1 {Ó}{{\'O}}1 {Ú}{{\'U}}1
  {à}{{\`a}}1 {è}{{\`e}}1 {ì}{{\`i}}1 {ò}{{\`o}}1 {ù}{{\`u}}1
  {À}{{\`A}}1 {È}{{\'E}}1 {Ì}{{\`I}}1 {Ò}{{\`O}}1 {Ù}{{\`U}}1
  {ä}{{\"a}}1 {ë}{{\"e}}1 {ï}{{\"i}}1 {ö}{{\"o}}1 {ü}{{\"u}}1
  {Ä}{{\"A}}1 {Ë}{{\"E}}1 {Ï}{{\"I}}1 {Ö}{{\"O}}1 {Ü}{{\"U}}1
  {â}{{\^a}}1 {ê}{{\^e}}1 {î}{{\^i}}1 {ô}{{\^o}}1 {û}{{\^u}}1
  {Â}{{\^A}}1 {Ê}{{\^E}}1 {Î}{{\^I}}1 {Ô}{{\^O}}1 {Û}{{\^U}}1
  {Ã}{{\~A}}1 {ã}{{\~a}}1 {Õ}{{\~O}}1 {õ}{{\~o}}1
  {œ}{{\oe}}1 {Œ}{{\OE}}1 {æ}{{\ae}}1 {Æ}{{\AE}}1 {ß}{{\ss}}1
  {ű}{{\H{u}}}1 {Ű}{{\H{U}}}1 {ő}{{\H{o}}}1 {Ő}{{\H{O}}}1
  {ç}{{\c c}}1 {Ç}{{\c C}}1 {ø}{{\o}}1 {å}{{\r a}}1 {Å}{{\r A}}1
  {€}{{\euro}}1 {£}{{\pounds}}1 {«}{{\guillemotleft}}1
  {»}{{\guillemotright}}1 {ñ}{{\~n}}1 {Ñ}{{\~N}}1 {¿}{{?`}}1
  {º}{{\textordmasculine}}1}

\lstset{opliterate=
   *{0}{{{\color{codered}0}}}1 {1}{{{\color{codered}1}}}1 
   {2}{{{\color{codered}2}}}1 {3}{{{\color{codered}3}}}1 
   {4}{{{\color{codered}4}}}1 {5}{{{\color{codered}5}}}1 
   {6}{{{\color{codered}6}}}1 {7}{{{\color{codered}7}}}1 
   {8}{{{\color{codered}8}}}1 {9}{{{\color{codered}9}}}1}

\DeclareCaptionType{code}[Código][ÍNDICE DE CÓDIGOS] % Se define el entorno "Código" (de forma que al introducir un fragmento de código en el documento aparezca como: Código 1.1: ...), y la lista con los distintos códigos ("Índice de códigos").
\counterwithin{code}{section} % Se numeran los códigos con respecto al capítulo en el que se encuentran.
% ---------------------------------------------- % 


% --------------- Bibliografía ----------------- %
% El manejo de la bibliografía se realiza mediante el paquete biblatex:
\usepackage[backend=bibtex, style=authoryear, sorting=nyt, citestyle=authoryear, maxcitenames=2, maxbibnames=5, giveninits=true, uniquename=init]{biblatex} 

% Los distintos parámetros que aparecen en la línea anterior corresponden a las siguientes características de la bibliografía:
% - style: la manera en la que aparecen las referencias en la bibliografía. En este caso se opta por "authoryear", pero existen múltiples estilos posibles que se resumen en la siguiente guía: https://www.overleaf.com/learn/latex/biblatex_bibliography_styles.
% - sorting: orden en el que aparecen las distintas referencias en la bibliografía. En este caso se opta por ordenarlas en primer lugar por el apellido del primer autor, en segundo lugar por el año de publicación, y por último por el título de la publicación (nyt=name-year-title)
% - citestyle: elementos y orden de dichos elementos de una referencia al citarla en el documento. En este caso se escoge "authoryear" para que aparezca en primer lugar el apellido del autor (o de los autores) y en segundo lugar el año de publicación. Existe gran variedad de opciones en cuanto al parámetro citestyle que se resumen en: https://www.overleaf.com/learn/latex/biblatex_citation_styles.
% maxcitenames: máximo número de autores que aparecen al citar una referencia en el documento. Al escoger un valor de 2 para este parámetro se pueden dar los siguientes casos: un único autor -> (autor, año), dos autores -> (autor 1 y/e autor 2, año), tres o más autores -> (autor 1 et al., año).
% maxbibnames: parámetro idéntico al anterior pero para la bibliografía en lugar de las citas.
% giveinits y uniquename: para mostrar únicamente las iniciales de los nombres de los autores.

% Se importa el paquete csquotes para citar las referencias a lo largo del documento:
\usepackage{csquotes} 

% Se realizan una serie de operaciones para adaptar la bibliografía al estilo deseado (coma entre autor y año al citar una referencia, idioma castellano, etc.):
\DeclareNameAlias{sortname}{family-given}
\renewcommand*{\nameyeardelim}{\addcomma\space}
\setlength\bibitemsep{\baselineskip}
\DefineBibliographyStrings{spanish}{%
  andothers = {et\addabbrvspace al\adddot}
}

\makeatletter

\newrobustcmd*{\parentexttrack}[1]{%
  \begingroup
  \blx@blxinit
  \blx@setsfcodes
  \blx@bibopenparen#1\blx@bibcloseparen
  \endgroup}

\AtEveryCite{%
  \let\parentext=\parentexttrack%
  \let\bibopenparen=\bibopenbracket%
  \let\bibcloseparen=\bibclosebracket}

\makeatother

\addbibresource{content/biblio.bib}
% ---------------------------------------------- % 

%%%%%%%%%%%% - INICIO DEL DOCUMENTO - %%%%%%%%%%%%

\begin{document} 

%%%%%%%%%%%%%%%%%%%%%%%%%%%%%%%%%%%%%%%%%%%%%%%%%%


%%%%%%%%%%%%%%%%%%% - PORTADA - %%%%%%%%%%%%%%%%%%

% Se comienza una página nueva sin formato (sin número de página y sin encabezado/pie de página), ya que sólo incorpora la la portada:
\newpage
\thispagestyle{empty}

% La portada se inserta mediante el comando \includepdf seguido del archivo PDF correspondiente (que se ajusta automáticamente a las dimensiones de la página):
\includepdf{Portada_TFG_Antonio_Dies.pdf}

%%%%%%%%%%%%%%%%%%%%%%%%%%%%%%%%%%%%%%%%%%%%%%%%%%

% Las páginas anteriores al contenido del TFG/TFM (previas a la introducción) suelen numerarse de forma distinta a las del cuerpo del informe, en este caso en números romanos:
\pagenumbering{roman}

%%%%%%%%%%%%%%%%%%%%%%%%%%%%%%%%%%%%%%%%%%%%%%%%%%


%%%%%%%%%%%%%$%%%%% - CITA - %%%%%%%%%%%%%%%%%%%%%
 
% Se comienza una página nueva sin formato (sin número de página y sin encabezado/pie de página), ya que sólo incorpora la cita:
\newpage
\thispagestyle{empty}

\begin{flushright} % Se alinea el texto en el lado derecho de la página.
\vspace*{5cm} % Se añade un espacio vertical de 5cm para situar la cita en ~1/3 de la página.

\textit{“La cita del trabajo iría aquí”} 

\medskip % Salto a la línea de tamaño medio (existen \smallskip, \medskip y \bigskip)
- El autor de la cita 

\end{flushright}

\afterpage{\blankpage} % Se añade una página en blanco después de la cita.

%%%%%%%%%%%%%%%%%%%%%%%%%%%%%%%%%%%%%%%%%%%%%%%%%%


%%%%%%%%%%%%% - AGRADECIMIENTOS - %%%%%%%%%%%%%%%%

% Se comienza una página nueva con formato plano (sin encabezado/pie de página pero con número de página):
\newpage
\thispagestyle{plain}

\section*{AGRADECIMIENTOS} % Se añade un asterisco a \section para que el título no esté numerado.
\addcontentsline{toc}{section}{AGRADECIMIENTOS} % Al utilizar \section* se ha de añadir manualmente el apartado al índice (Table Of Contents, TOC).

Agradezco a \dots

Gracias a \dots

A \dots \ por \dots

\afterpage{\blankpage} % Se añade una página en blanco después de los agradecimientos.

%%%%%%%%%%%%%%%%%%%%%%%%%%%%%%%%%%%%%%%%%%%%%%%%%%


%%%%%%%%%%%%%% - RESUMEN EJECUTIVO - %%%%%%%%%%%%%

\newpage
\section*{RESUMEN} % Se añade un asterisco a \section para que el título no esté numerado.
\markright{RESUMEN} % Al utilizar \section* se ha de añadir manualmente el título del apartado al encabezado.
\addcontentsline{toc}{section}{RESUMEN} % Al utilizar \section* se ha de añadir manualmente el apartado al índice (Table Of Contents, TOC).

Este documento constituye una guía (que sirve a su vez de plantilla) para la elaboración de informes de TFG o TFM en \LaTeX. No pretende abarcar todas y cada una de las funcionalidades que ofrece \LaTeX \ (¡las posibilidades son prácticamente infinitas!) pero sí tratar los aspectos fundamentales para la elaboración de un documento utilizando esta indispensable herramienta. Además de los elementos básicos de cualquier informe (índice, tablas, ecuaciones, bibliografía, etc.), esta guía incluye ``tutoriales'' y plantillas para algunos de los elementos presentes en todo (o casi todo) informe de TFG o TFM (como son el diagrama de Gantt o la EDP). 

\textbf{Nota:} se ha tratado de explicar con detalle la mayor parte de elementos presentes en el documento, ya sea por medio de los capítulos y apartados que lo conforman o mediante explicaciones bajo la forma de comentarios en el código \LaTeX. Es especialmente importante examinar con atención el preámbulo de dicho código, ya que en él se llevan a cabo muchas de las operaciones esenciales que dan forma al documento.

\afterpage{\blankpage} % Se añade una página en blanco después del resumen.

%%%%%%%%%%%%%%%%%%%%%%%%%%%%%%%%%%%%%%%%%%%%%%%%%%


%%%%%%%%%%%%%%%%%%% - ÍNDICE - %%%%%%%%%%%%%%%%%%%

\newpage

\renewcommand*\contentsname{ÍNDICE} % Se modifica el nombre por defecto de la "Table Of Contents" (tabla de contenidos, índice) para pasar a llamarla "ÍNDICE".

\tableofcontents % Se genera el índice de contenidos del documento que incorpora todos los títulos de \section, \subsection y \subsubsection (y también \paragraph, ver capítulo 1), así como los títulos añadidos con \addcontentsline (como el resumen ejecutivo, por ejemplo).

\afterpage{\blankpage} % Se añade una página en blanco después del índice.


%%%%%%%%%%%%%%%%%%%%%%%%%%%%%%%%%%%%%%%%%%%%%%%%%%

% Se inicia una nueva página, y se restablece la numeración de las páginas, utilizando esta vez el sistema de numeración estándar (1, 2, 3, 4, ...)
\newpage
\pagenumbering{arabic}

%%%%%%%%%%%%%%%%%%%%%%%%%%%%%%%%%%%%%%%%%%%%%%%%%%


\section{INTRODUCCIÓN} \label{sec:introduccion}

\subsection{Justificación}

Actualmente, el mundo atraviesa una crisis energética global desencadenada en el año 2021 principalmente por la súbita recuperación económica tras la pandemia y agravada gradualmente hasta consumarse tras la invasión rusa de Ucrania en febrero de 2022. El precio del gas natural alcanzó máximos históricos, aumentando consecuentemente en muchos casos el coste de la electricidad en general. Familias, empresas e industrias se han visto gravemente afectadas, llevando a diversos países en camino de una fuerte recesión económica. Consecuentemente, la reducción de los costes energéticos se convierte en una de las principales prioridades de empresas y ciudadanos, y la independencia energética, la
garantía de suministro y la lucha contra el cambio climático adquieren una gran importancia en el debate público de gran cantidad de países (\cite{crisis_energetica_iea}). 

Frente a esta situación, la energía nuclear está tomando cada vez más relevancia en muchos países, considerándose un factor clave para conseguir los grandes desafíos políticos, económicos y climáticos  a los que se enfrenta la sociedad actual en un escenario tan complicado. Numerosos países han optado por ampliar su parque nuclear existente, muchos han decidido alargar la vida de sus reactores nucleares actualmente en operación y algunos han comenzado a construir sus primeras centrales nucleares. 

\begin{table}[h]
    \resizebox{\textwidth}{!}{%
    \begin{tabular}{|cc|cc|cc|}
    \hline
    \rowcolor[HTML]{ECF4FF} 
    \multicolumn{2}{|l|}{\cellcolor[HTML]{ECF4FF}\textbf{Generación de electricidad nuclear}} &
      \multicolumn{2}{l|}{\cellcolor[HTML]{ECF4FF}\textbf{Reactores en operación}\tablefootnote{\textbf{En operación:} Conectados a la red.}} &
      \multicolumn{2}{l|}{\cellcolor[HTML]{ECF4FF}\textbf{Reactores en construcción}\tablefootnote{\textbf{En construcción:} primer hormigón vertido para el reactor.}} \\ \hline
    \rowcolor[HTML]{FFFFFF} 
    \multicolumn{1}{|c|}{\cellcolor[HTML]{FFFFFF}\textbf{9,8}} &
      2.808 TWh &
      \multicolumn{1}{c|}{\cellcolor[HTML]{FFFFFF}\textbf{436}} &
      392.114 MWe &
      \multicolumn{1}{c|}{\cellcolor[HTML]{FFFFFF}{\color[HTML]{000000} \textbf{62}}} &
      {\color[HTML]{000000} 69.279 MWe} \\ \hline
    \rowcolor[HTML]{ECF4FF} 
    \multicolumn{2}{|c|}{\cellcolor[HTML]{ECF4FF}\textbf{Reactores planificados}\tablefootnote{\textbf{Planificados:} Aprobaciones, financiamiento o compromiso en vigor. Se espera que estén en funcionamiento en los próximos 15 años.}} &
      \multicolumn{2}{c|}{\cellcolor[HTML]{ECF4FF}\textbf{Reactores propuestos}\tablefootnote{\textbf{Propuestos:} Programa específico o propuestas de sitio; tiempo muy incierto.}} &
      \multicolumn{2}{c|}{\cellcolor[HTML]{ECF4FF}\textbf{OLP aprobada}\tablefootnote{\textbf{\acrfull{olp} aprobada:} Autorización a operar más allá de los 40 años. En Estados Unidos, la mayoría de reactores tiene licencia para operar a 60 años y 6 tienen permiso para operar hasta los 80.}} \\ \hline
    \multicolumn{1}{|c|}{\cellcolor[HTML]{FFFFFF}\textbf{110}} &
      \cellcolor[HTML]{FFFFFF}112.877 MWe &
      \multicolumn{1}{c|}{\cellcolor[HTML]{FFFFFF}\textbf{333}} &
      \cellcolor[HTML]{FFFFFF}366.652 MWe &
      \multicolumn{2}{c|}{\textbf{191}} \\ \hline
    \end{tabular}%
    }
    \caption{Resumen de la situación actual de la energía nuclear en el mundo (\cite{world_nuclear_power_reactors}).}
    \label{tab:situacion_nuclear_mundial}
    \end{table}

    \begin{wrapfigure}{r}{0.52\textwidth}
      \vspace{-0.5cm}
      \centering
      \includegraphics[width=0.52\textwidth]{content/figures/global_smr_projects2.png}
      \caption{\acrshortpl{smr} en el mundo (\cite{iea_global_smr_projects}).}
      \label{fig:global_smr_projects}
      \vspace{-1cm}
    \end{wrapfigure}

    En este contexto, se ha incrementado muy considerablemente el interés por los reactores modulares pequeños, ampliamente conocidos como \textbf{\emph{\acrfullpl{smr}}}. Se trata de una tecnología avanzada de menor escala que la convencional que ofrece grandes ventajas en lo que a coste, tiempo de construcción, seguridad y versatilidad se refiere. Por consiguiente, múltiples instituciones públicas y privadas están participando activamente en los esfuerzos encaminados a hacer prosperar esta tecnología, existiendo más de 80 diseños de \acrshortpl{smr} comerciales que se están desarrollando en todo el mundo (\cite{smr_oiea}).




    
\newpage 

Este creciente empuje de la industria nuclear está contribuyendo a un aumento de profesionales especializados en este sector y, paralelamente, a una creciente necesidad de futuros profesionales nucleares. En este contexto y frente a los grandes avances tecnoloógicos desarrollados actualmente, cobran una especial importancia los \textbf{simuladores} empleados tanto en la profesión como en la formación de operadores, técnicos e ingenieros nucleares. Existen múltiples simuladores virtuales y físicos desarrollados por diversas instituciones y empresas que permiten enfrentarse a las condiciones de operación, maniobras y accidentes que pueden suceder en una central nuclear. La Escuela Técnica Superior de Ingenieros Industriales de Madrid (ETSII - UPM) tiene a su disposición el \acrfull{sgiz}, con el cual se trabajará en el presente proyecto para profundizar en el estudio de la operación de las centrales nucleares y, en concreto, en la operación de un \acrshort{smr}, debido a las grandes similitudes que el simulador en cuestión presenta con respecto a esta innovadora tecncología.

\subsection{Objetivos}

El principal objetivo de este trabajo fin de grado es \textbf{conocer en profundidad el funcionamiento de un \acrshort{smr}; sus sistemas de seguridad, protección y control, su modo de operación y su respuesta frente a diversos sucesos adversos}. Para ello, tras la necesaria documentación sobre este tipo de reactores, se procederá a simular la operación normal y diversos transitorios de una central nuclear muy similar a un \acrshort{smr} mediante el \acrshort{sgiz} de la Escuela.

Asimismo, existen paralelamente diversos objetivos secundarios. En primer lugar, familiarizarse con el tipo de software empleado en los simuladores del ámbito nuclear. En segundo lugar, conocer el estado del arte, las características, las grandes ventajas y los desafíos de la tecnología de los \acrshortpl{smr}. Por último, implementar las simulaciones realizadas al programa de prácticas de la asignatura de Tecnologías Avanzadas en Reactores Nucleares del Máster en Ciencia y Tecnología Nuclear impartido en la ETSII.

\subsection{Metodología}

El desarrollo de este trabajo comprende dos grandes bloques: un marco teórico y un marco práctico.

El \textbf{marco teórico} incluye, en primer lugar, un detallado estudio del estado del arte de los \emph{Small Modular Reactors}. En segundo lugar, se incorpora un análisis del funcionamiento y abanico de posibilidades que ofrecen los simuladores ---en concreto, el \acrshort{sgiz}---. Por último, se hace un estudio de las similitudes que presenta el simulador en cuestión con un \acrshort{smr}. 

El \textbf{marco práctico} se fundamenta en la simulación de la operación normal y de distintos transitorios en el \acrshort{sgiz}, con el fin de comprender mejor el funcionamiento de las centrales nucleares y, en concreto, de las de menor escala, como lo son los \acrshortpl{smr}. Como valor añadido, se plantea la posible implementación de las simulaciones realizadas en el programa de prácticas del máster.
%%%%%%%%%%%%%%%% - BIBLIOGRAFÍA - %%%%%%%%%%%%%%%%

\newpage

El formato elegido para la bibliografía es APA (el recomendable para informes de TFG/TFM), tanto para las referencias a lo largo del documento como para el apartado de bibliografía. El conjunto de operaciones realizadas para establecer el formato de la bibliografía se puede consultar en el preámbulo del documento (en el que se describen algunos de sus parámetros básicos como el contenido de las referencias, el número de autores por cita, etc.).

Citar una referencia es sencillo, basta con utilizar el comando \textbackslash\texttt{cite} seguido del nombre de la referencia correspondiente (el nombre utilizado en el archivo \textit{.bib}, que es esencial cargar en el directorio de trabajo y cuyas principales características pueden consultarse en \url{https://en.wikipedia.org/wiki/BibTeX}), por ejemplo:

\begin{itemize}
    \item \textit{The Art of Electronics} constituye un fantástico manual (plagado de ejemplos prácticos y explicaciones tangibles) para aprender electrónica, siendo su tercera edición la versión más completa (\cite{horowitz2015}).
    \item \textit{The Loudspeaker Design Cookbook} (\cite{dickson2007}) es probablemente la guía más completa en cuanto a acústica aplicada al diseño de sistemas de sonido, abarcando desde conceptos teóricos de electroacústica hasta planos para la construcción de sistemas de sonido caseros.
    \item \textit{Les fous du son} (\cite{dewilde2016}) es un relato cuidadosamente escrito y documentado sobre la historia de los sintetizadores desde Edison hasta nuestros días, pasando por los inventos más inverosímiles como las Ondas Martenot o el Trautonium.
    \item En su artículo de 2003 (\cite{wang2003}), el co-fundador de Shazam describe el funcionamiento de su algoritmo de búsqueda para archivos de audio.
\end{itemize}

% Se genera la bibliografía mediante el comando \printbibliography (en ella aparecen únicamente las referencias citadas a lo largo del documento):
\appto{\bibsetup}{\sloppy}
\printbibliography[heading=bibintoc, title=BIBLIOGRAFÍA] % el argumento "title" puede modificarse indicando el título que convenga (bibliografía, referencias, etc.).

%%%%%%%%%%%%%%%%%%%%%%%%%%%%%%%%%%%%%%%%%%%%%%%%%%  
%%%%%%%%%%%%%%%%%%% - ANEXOS - %%%%%%%%%%%%%%%%%%%

\newpage

\section*{ANEXOS} \label{sec:anexos} % Se añade un asterisco a \section para que el título no esté numerado.
\addcontentsline{toc}{section}{ANEXOS} % Al utilizar \section* se ha de añadir manualmente el apartado al índice (Table Of Contents, TOC).
\markright{ANEXOS} % Al utilizar \section* se ha de añadir manualmente el título del apartado al encabezado.

\renewcommand{\thesubsection}{\Alph{subsection}} % Se numeran los anexos con letras del alphabeto en lugar de números.
% Se indica que las tablas, figuras y códigos se numeran con el código del anexo (A, B, C, ...) seguido del número de tabla, figura o código dentro del anexo (tabla A.2, figura C.1, etc.)
\renewcommand{\thetable}{\Alph{subsection}.\arabic{table}}
\renewcommand{\thefigure}{\Alph{subsection}.\arabic{figure}}
\renewcommand{\thecode}{\Alph{subsection}.\arabic{code}}

% ---------------- Primer anexo ---------------- %
\subsection{Primer anexo} \label{sec:anexo1}

Contenido del primer anexo (texto, tablas, figuras, códigos, etc.)

% ---------------- Segundo anexo --------------- %
\newpage
\subsection{Segundo anexo} \label{sec:anexo2}

%%%%%%%%%%%%%%%%%%%%%%%%%%%%%%%%%%%%%%%%%%%%%%%%%%


%%%%%%%%%%%%%% - ÍNDICE DE TABLAS - %%%%%%%%%%%%%%

\newpage

\renewcommand{\listtablename}{ÍNDICE DE TABLAS} % Se define el nombre del índice de tablas.
\listoftables % Se genera automáticamente el índice con las distintas tablas del documento (entorno \table o \longtable).
\addcontentsline{toc}{section}{ÍNDICE DE TABLAS} % Se añade manualmente el apartado al índice (Table Of Contents, TOC).

%%%%%%%%%%%%%%%%%%%%%%%%%%%%%%%%%%%%%%%%%%%%%%%%%%


%%%%%%%%%%%%% - ÍNDICE DE FIGURAS - %%%%%%%%%%%%%%

\newpage

\renewcommand{\listfigurename}{ÍNDICE DE FIGURAS} % Se define el nombre del índice de figuras.
\listoffigures % Se genera automáticamente el índice con las distintas figuras del documento (entorno \figure).
\addcontentsline{toc}{section}{ÍNDICE DE FIGURAS} % Se añade manualmente el apartado al índice (Table Of Contents, TOC).

%%%%%%%%%%%%%%%%%%%%%%%%%%%%%%%%%%%%%%%%%%%%%%%%%%


%%%%%%%%%%%%%% - ÍNDICE DE CÓDIGOS - %%%%%%%%%%%%%

\newpage

\listofcodes % Se genera automáticamente el índice con los distintos códigos del documento (entorno \code).
\addcontentsline{toc}{section}{ÍNDICE DE CÓDIGOS} % Se añade manualmente el apartado al índice (Table Of Contents, TOC).

\afterpage{\blankpage} % Se añade una página en blanco después del índice de códigos.

%%%%%%%%%%%%%%%%%%%%%%%%%%%%%%%%%%%%%%%%%%%%%%%%%%



%%%%%%%%%%%%%%%%%%%%%%%%%%%%%%%%%%%%%%%%%%%%%%%%%%


%%%%%%%%%%%%%% - FIN DEL DOCUMENTO - %%%%%%%%%%%%%

\end{document}

%%%%%%%%%%%%%%%%%%%%%%%%%%%%%%%%%%%%%%%%%%%%%%%%%%
