
%%%%%%%%%%%%%%%%% - PREÁMBULO - %%%%%%%%%%%%%%%%% 

% ------------------ Página -------------------- %
% Se define el tamaño de las páginas, indicando el tamaño de los márgenes superior e inferior ("top" y "bottom"), e izquierdo y derecho ("left" y "right"):
\usepackage[top=2.5cm,bottom=2.5cm,left=2.5cm,right=2.5cm]{geometry}
% Se inserta el comando \raggedbottom para evitar que LaTeX rellene con espacios en blanco aquellas páginas que no alberguen suficiente contenido como para rellenarlas de forma "natural":
\raggedbottom 
% ---------------------------------------------- %


% ------------- Paquetes generales ------------- %
% Se importan distintos paquetes de propósito general:
\usepackage[utf8]{inputenc}
\usepackage[spanish,es-tabla]{babel}
\usepackage{float}
\usepackage{caption}
% ---------------------------------------------- %


% ------------ Paquetes específicos ------------ %
% Se importan distintos paquetes que será utilizados en momentos concretos del documento: 
\usepackage{pdfpages} % Para insertar la portada en formato PDF.
\usepackage{amssymb} % Para símbolos matemáticos.
\usepackage{bm} % Para negrita en símbolos matemáticos.
\usepackage{amsmath} % Para el entorno "split".
\usepackage[hidelinks]{hyperref} % Para urls.
\usepackage{longtable} % Para tablas largas.
\usepackage{graphicx} % Para insertar imágenes.
\usepackage{wrapfig} % Para posicionar imágenes alrededor del texto.
\usepackage{fontawesome5} % Para utilizar iconos de "fontawesome".
\usepackage{pdflscape}  % Para colocar páginas en formato apaisado.
\usepackage[T1]{fontenc}
\usepackage{textcomp}
\usepackage{lmodern} % Soluciona el problema de la mala resolución
\usepackage{url}
\usepackage{colortbl}
\usepackage{longtable}
\usepackage{array}
\usepackage{tablefootnote}
\usepackage{wrapfig}
% ---------------------------------------------- %


% ---------------- Numeración ------------------ %
\counterwithin{table}{section} % Se numeran las tablas con respecto al capítulo en el que se encuentran.
\counterwithin{figure}{section} % Se numeran las figuras con respecto al capítulo en el que se encuentran.
\counterwithin{equation}{section} % Se numeran las ecuaciones con respecto al capítulo en el que se encuentran.
% ---------------------------------------------- %


% ------------- Página en blanco ----------------%
% Se define un comando (\blankpage) para insertar una página totalmente en blanco (sin número de página, encabezado y pie de página):
\usepackage{afterpage}
\newcommand\blankpage{%
    \null
    \thispagestyle{empty}%
    \newpage}
% ---------------------------------------------- %    


% ----------- Formato de los párrafos -----------%
% Se define el formato de los párrafos:
\setlength{\parindent}{0pt} % Se elimina la sangría en comienzo de párrafo (0pt).
\setlength{\parskip}{1em} % Se define el espacio entre dos párrafos (1em).
% ---------------------------------------------- %    

% -------------- Título adicional -------------- %
% Se añade una profundidad adicional a los títulos (profundidad 4):
\usepackage{titlesec}
\titlespacing{\subsubsection}{1.5pc}{1.5ex plus .1ex minus .2ex}{1.5ex plus .2ex} 
\setcounter{secnumdepth}{4} % Se fija en 4 la profundidad de numeración de títulos.
\setcounter{tocdepth}{4} % Se fija en 4 la profundidad de títulos incluidos en el índice.
% Se modifica el formato de \paragraph (título de profundidad 4) para adaptarlo al formato del resto de títulos:
\titleformat{\paragraph}
{\normalfont\normalsize\bfseries}{\theparagraph}{1em}{}
\titlespacing{\paragraph}{3pc}{1.5ex plus .1ex minus .2ex}{1.5ex plus .2ex} 
% ---------------------------------------------- %    


% --------- Encabezado y pie de página -------- %
% El encabezado y pie de página forman parte del paquete fancyhdr:
\usepackage{fancyhdr}
\fancyhf{}
\pagestyle{fancy}

% Para solucionar error "headheight is too small":%
\setlength{\headheight}{14.6pt}
\addtolength{\topmargin}{-0.6pt}

% Se ajusta el tamaño de fuente para el encabezado y pie de página (9pt)
\fancyhf{\fontsize{2}{14}\selectfont}

% Contenido del encabezado (\fancyhead):
\fancyhead[RO]{Simulación de operación de una central nuclear} % Texto que se coloca en el encabezado de las páginas impares (O -> 'Odd', o impar) a la izquierda (R -> 'Odd')
\fancyhead[LE]{\nouppercase{\rightmark}} % Texto que se coloca en el encabezado de las páginas pares (E -> 'Even', o par) a la izquierda (L -> 'Left'). \rightmark se utiliza para insertar automáticamente el título de la sección correspondiente, y \nouppercase para que no aparezca todo en mayúsculas (formato por defecto de \rightmark).

% Contenido del pie de página (\fancyfoot):
\fancyfoot[RE]{Escuela  Técnica  Superior  de  Ingenieros  Industriales  (UPM)} % Texto que se coloca en el pie de página de las páginas pares (E -> 'Even', o par) a la derecha (R -> 'Right')
\fancyfoot[LO]{Antonio Dies Beneytez} % Texto que se coloca en el pie de página de las páginas impares (O -> 'Odd', o impar) a la izquierda (L -> 'Left')
\fancyfoot[LE,RO]{\thepage} % El número de página (\thepage) se coloca a la izquierda en las páginas pares y a la derecha en las impares.

% Se indica que sólo se quiere incorporar en \rightmark (utilizado más arriba) el título de la sección (y no de las subsecciones, subsubsecciones, etc.):
\renewcommand{\sectionmark}[1]{\markright{\thesection. #1}}
\renewcommand{\subsectionmark}[1]{}

% Formato de la línea de separación horizontal:
\renewcommand{\headrulewidth}{0.5pt} % Ancho de la línea del encabezado.
\renewcommand{\footrulewidth}{0.5pt} % Ancho de la línea del pie de página.
% ---------------------------------------------- % 


% --------------- Bibliografía ----------------- %
% El manejo de la bibliografía se realiza mediante el paquete biblatex:
\usepackage[backend=bibtex, style=authoryear, sorting=nyt, citestyle=authoryear, maxcitenames=2, maxbibnames=5, giveninits=true, uniquename=init]{biblatex} 

% Los distintos parámetros que aparecen en la línea anterior corresponden a las siguientes características de la bibliografía:
% - style: la manera en la que aparecen las referencias en la bibliografía. En este caso se opta por "authoryear", pero existen múltiples estilos posibles que se resumen en la siguiente guía: https://www.overleaf.com/learn/latex/biblatex_bibliography_styles.
% - sorting: orden en el que aparecen las distintas referencias en la bibliografía. En este caso se opta por ordenarlas en primer lugar por el apellido del primer autor, en segundo lugar por el año de publicación, y por último por el título de la publicación (nyt=name-year-title)
% - citestyle: elementos y orden de dichos elementos de una referencia al citarla en el documento. En este caso se escoge "authoryear" para que aparezca en primer lugar el apellido del autor (o de los autores) y en segundo lugar el año de publicación. Existe gran variedad de opciones en cuanto al parámetro citestyle que se resumen en: https://www.overleaf.com/learn/latex/biblatex_citation_styles.
% maxcitenames: máximo número de autores que aparecen al citar una referencia en el documento. Al escoger un valor de 2 para este parámetro se pueden dar los siguientes casos: un único autor -> (autor, año), dos autores -> (autor 1 y/e autor 2, año), tres o más autores -> (autor 1 et al., año).
% maxbibnames: parámetro idéntico al anterior pero para la bibliografía en lugar de las citas.
% giveinits y uniquename: para mostrar únicamente las iniciales de los nombres de los autores.

% Se importa el paquete csquotes para citar las referencias a lo largo del documento:
\usepackage{csquotes} 

% Se realizan una serie de operaciones para adaptar la bibliografía al estilo deseado (coma entre autor y año al citar una referencia, idioma castellano, etc.):
\DeclareNameAlias{sortname}{family-given}
\renewcommand*{\nameyeardelim}{\addcomma\space}
\setlength\bibitemsep{\baselineskip}
\DefineBibliographyStrings{spanish}{%
  andothers = {et\addabbrvspace al\adddot}
}

\makeatletter

\newrobustcmd*{\parentexttrack}[1]{%
  \begingroup
  \blx@blxinit
  \blx@setsfcodes
  \blx@bibopenparen#1\blx@bibcloseparen
  \endgroup}

\AtEveryCite{%
  \let\parentext=\parentexttrack%
  \let\bibopenparen=\bibopenbracket%
  \let\bibcloseparen=\bibclosebracket}

\makeatother

\addbibresource{content/sections/biblio.bib}

% --------------- Abreviaturas, unidades y acrónimos ----------------- %

\usepackage{glossaries}
\makenoidxglossaries

\newglossaryentry{simuladores}
{
    name=Simuladores,
    text={simuladores},
    description={Los simuladores son... Esto es un ejemplo para el glosario}
}

\newacronym{smr}{SMR}{\emph{Small Modular Reactor}}
\newacronym{sgiz}{SGIZ}{Simulador Gráfico Interactivo de Zorita}
\newacronym{pwr}{PWR}{\emph{Pressure Water Reactor}}
\newacronym{enresa}{Enresa}{Empresa Nacional de Residuos Radioactivos, S.A.}
\newacronym{ati}{ATI}{Almacén Temporal Individualizado}
\newacronym{oiea}{OIEA}{Organismo Internacional de Energía Atómica}
\newacronym{idi}{I+D+I}{Investigación, Desarrollo e Innovación}
\newacronym{svfc}{SVFC}{Sistema de Venteo Filtrado de Contención}
\newacronym{nsss}{NSSS}{\emph{Nuclear Steam Supply System}}
\newacronym{mmr}{MMR}{\emph{Micro Modular Reactor}}
\newacronym{genIV}{Gen IV}{Generación IV}
\newacronym{lwr}{LWR}{\emph{Light Water Reactor}}
\newacronym{hwr}{HWR}{\emph{Heavy Water Reactor}}
\newacronym{htgr}{HTGR}{\emph{Hight Temperature Gas-cooled Reactor}}
\newacronym{triso}{TRISO}{Combustible tri-isotrópico}
\newacronym{msr}{MSR}{\emph{Molten Salt Reactor}}
\newacronym{lmfr}{LMFR}{\emph{Liquid Metal cooled Fast Reactor}}
\newacronym{olp}{OLP}{Operación a Largo Plazo}
\newacronym{loca}{LOCA}{\emph{Loss-Of-Coolant Accident}}
\newacronym{epz}{EPZ}{\emph{Emergency Planning Zone}}

% ---------------------------------------------- % 