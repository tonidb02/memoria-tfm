\section{INTRODUCCIÓN} \label{sec:introduccion}

\subsection{Antecedentes}
texto
\subsection{Justificación}

Actualmente, el mundo atraviesa una crisis energética global desencadenada en el año 2021 principalmente por la súbita recuperación económica tras la pandemia y agravada gradualmente hasta consumarse tras la invasión rusa de Ucrania en febrero de 2022. El precio del gas natural alcanzó máximos históricos, aumentando consecuentemente en muchos casos el coste de la electricidad en general. Familias, empresas e industrias se han visto gravemente afectadas, llevando a diversos países en camino de una fuerte recesión económica. Consecuentemente, la reducción de los costes energéticos se convierte en una de las principales prioridades de empresas y ciudadanos, y la independencia energética, la
garantía de suministro y la lucha contra el cambio climático adquieren una gran importancia en el debate público de gran cantidad de países (\cite{crisis_energetica_iea}). 

Frente a esta situación, la energía nuclear está tomando cada vez más relevancia en muchos países, considerándose un factor clave para conseguir los grandes desafíos políticos, económicos y climáticos  a los que se enfrenta la sociedad actual en un escenario tan complicado. Numerosos países han optado por ampliar su parque nuclear existente, muchos han decidido alargar la vida de sus reactores nucleares actualmente en operación y algunos han comenzado a construir sus primeras centrales nucleares. 

En este contexto, se ha incrementado muy considerablemente el interés por los reactores modulares pequeños, ampliamente conocidos como \textbf{\emph{\acrfull{smr}}}. Se trata de una tecnología avanzada de menor escala que la convencional que ofrece grandes ventajas en lo que a coste, tiempo de construcción, seguridad y versatilidad se refiere. Por consiguiente, múltiples instituciones públicas y privadas están participando activamente en los esfuerzos encaminados a hacer prosperar esta tecnología, existiendo más de 70 diseños de SMR comerciales que se están desarrollando en todo el mundo (\cite{smr_oiea}).

Este creciente empuje de la industria nuclear está contribuyendo a un aumento de profesionales especializados en este sector y, paralelamente, a una creciente necesidad de futuros profesionales nucleares. En este contexto y frente a los grandes avances tecnoloógicos desarrollados actualmente, cobran una especial importancia los \textbf{\gls{simuladores}} empleados tanto en la profesión como en la formación de operadores, técnicos e ingenieros nucleares. Existen múltiples simuladores virtuales y físicos desarrollados por diversas instituciones y empresas que permiten enfrentarse a las condiciones de operación, maniobras y accidentes que pueden suceder en una central nuclear. La Escuela Técnica Superior de Ingenieros Industriales de Madrid (ETSII - UPM) tiene a su disposición el SGIZ (Simulador Gráfico Interactivo de Zorita), con el cual se trabajará en el presente proyecto para profundizar en el estudio de la operación de las centrales nucleares y, en concreto, en la operación de un SMR, debido a las grandes similitudes que el simulador en cuestión presenta con respecto a esta innovadora tecncología.

\subsection{Objetivos}
texto
\subsection{Metodología}

El desarrollo de este trabajo comprende dos grandes bloques: un marco teórico y un marco práctico.

El \textbf{marco teórico} incluye, en primer lugar, un detallado estudio del estado del arte de los \emph{Small Modular Reactors}. En segundo lugar, se incorpora un análisis del funcionamiento y abanico de posibilidades que ofrecen los simuladores ---en concreto, el SGIZ---. Por último, se hace un estudio de las similitudes que presenta el simulador en cuestión con un SMR. 

El \textbf{marco práctico} se fundamenta en la simulación de la operación normal y de distintos transitorios en el SGIZ, con el fin de comprender mejor el funcionamiento de las centrales nucleares y, en concreto, de las de menor escala, como lo son los SMRs. Como valor añadido, se plantea la posible implementación de las simulaciones realizadas en el programa de prácticas de la asignatura de Tecnologías Avanzadas en Reactores Nucleares del Máster en Ciencia y Tecnología Nuclear impartido en la ETSII.