\newpage
\section{SIMULADORES} \label{simuladores}

\subsection{Simuladores de SMRs}

IAEA - Simulador SMR: https://www.iaea.org/topics/nuclear-power-reactors/nuclear-reactor-simulators-for-education-and-training/integral-pressurized-water-reactor-simulator

¿Cómo ayudan los simuladores a los reactores SMR?\newline
(https://www.tecnatom.es/blog/smr-reactores-modulares-pequenos-el-futuro-de-la-energia-nuclear/)
Un simulador permite la familiarización con el diseño, con la operación normal, así como en la operación en condiciones anormales de operación como en la evolución y mitigación de diversas secuencias accidentales, mejorando así tanto la eficiencia como la seguridad en la operación de los SMR.

En Tecnatom llevamos décadas ofreciendo nuestras soluciones de simulación en el sector nuclear a nivel mundial y, concretamente, hemos tenido algunas experiencias relacionadas con los reactores SMR. Una de las más destacadas ha sido el desarrollo para la Agencia Internacional de la Energía Atómica (IAEA) de su simulador de principios básicos de SMR de agua a presión integrado (iPWR) genérico.

Un SMR iPWR consiste básicamente en un reactor de agua a presión con una potencia de 45 MW eléctricos, sin lazos primarios y con los generadores de vapor y el propio presionador integrados en la estructura de la vasija. El alcance del simulador incluye el ciclo agua-vapor para la generación de vapor (150 MW térmicos), los sistemas de seguridad y mitigación de accidentes, así como la contención.

Los SMR tendrán un papel fundamental en la transición energética y para ello será decisivo que éstos operen de manera segura y eficiente, para lo que los simuladores de entrenamiento serán de gran utilidad.

\subsection{La Central Nuclear José Cabrera y el SGIZ}

Tal y como se ha comentado en la introducción del proyecto, se pretende profundizar en el conocimiento de los \emph{\acrlongpl{smr}} mediante la simulación de la operación de una central nuclear con un comportamiento muy similar al de un \acrshort{smr}: la Central Nuclear de José Cabrera, también conocida como Central Nuclear de Zorita.

\subsubsection{La Central Nuclear de Zorita}

La Central Nuclear José Cabrera está situada en Almonacid de Zorita (Guadalajara) y fue la primera central nuclear española en operación.

Se trata de un \textbf{reactor de agua a presión (\acrshort{pwr})} de un lazo\footnote{Los lazos son los circuitos acoplados a la vasija del reactor que cuentan con un generador de vapor y una bomba para impulsar al refrigerante. El circuito primario de las centrales nucleares convencionales de gran escala suele contar con una sola vasija y un solo presionador, junto con 3 o 4 lazos (cada uno con una bomba y un generador de vapor).} diseñado por Westinghouse. Este tipo de reactores emplea agua ligera como refrigerante y moderador, trabajando con neutrones en el rango térmico para maximizar la tasa de fisión del $U^{235}$, presente en un 3,3\% ---en el caso de la central nuclear de Zorita--- del Uranio Enriquecido (UE) empleado como combustible.

La central comenzó a construirise en julio de 1965. Tras 36 meses de construcción, el 31 de marzo de 1968 se realizó la prueba funcional en caliente y en junio de ese mismo año se hizo la carga del núcleo, alcanzando su primera criticidad. La red eléctrica española comenzó a recibir los primeros kilovatios-hora de origen nuclear el 14 de julio de 1968 y, desde entonces, se inició un aumento gradual de potencia que llevó a la explotación comercial de la central, con una \textbf{potencia eléctrica instalada de 160 MWe} y una \textbf{potencia térmica de 510 MWt}. El titular de su explotación fue Unión Fenosa Generación (actualmente Naturgy). Durante 39 años de operación comercial produjo 36.515 millones de kWh, empleando a 300 trabajadores de forma directa y a unos 6.000 de forma indirecta.

\begin{figure}[h]
    \centering
    \includegraphics[width=\textwidth]{content/figures/zorita.jpg}
    \caption{Central Nuclear José Cabrera (\cite{sne_recursos_prensa}).}
    \label{fig:zorita}
\end{figure}

El cese definitivo de explotación de la central fue declarado por el Ministerio de Industria, Turismo y Comercio mediante la Orden Ministerial del 20 de abril de 2006. El titular de las actividades de desmantelamiento de la central es \acrshort{enresa}, por lo que se tubo que cambiar la titularidad de la instalación de Gas Natural Fenosa a \acrshort{enresa}. El 11 de febrero de 2010, \acrshort{enresa} inició el desmantelamiento de la instalación.

La alternativa escogida fue un desmantelamiento total e inmediato en un horizonte temporal de unos ocho años. Las actividades de desmantelamiento comprenden también la segmentación mediante técnicas de corte especiales de grandes sistemas y componentes radiológicamente significativos, como el generador de vapor y la vasija del reactor, así como la descontaminación y demolición de edificios y la restauración final del emplazamiento. 

Actualmente, la instalación se encuentra en la fase de restauración, en la cual se pretende devolver al emplazamiento a sus condiciones previas a la construcción de la central, saneando todo el terreno en cuestión. Los elementos de combustible irradiado de la central se están almacenando temporalmente en el \acrfull{ati} de la instalación, en el que también se han depositado algunos residuos generados durante el desmantelamiento (\cite{enresa_desmantelamiento_zorita}).

En la siguiente tabla se resume lo más importante de lo expuesto anteriormente sobre la central nuclear de Zorita:

\begin{table}[h]
    \resizebox{\textwidth}{!}{%
    \begin{tabular}{|cc|cl|cc|}
    \hline
    \rowcolor[HTML]{ECF4FF} 
    \multicolumn{2}{|c|}{\cellcolor[HTML]{ECF4FF}\textbf{Localización}} &
      \multicolumn{2}{c|}{\cellcolor[HTML]{ECF4FF}\textbf{Propiedad}} &
      \multicolumn{1}{c|}{\cellcolor[HTML]{ECF4FF}\textbf{Titular}} &
      \textbf{Tipo} \\ \hline
    \multicolumn{2}{|c|}{Almonacid de Zorita (Guadalajara)} & \multicolumn{2}{c|}{Unión Fenosa (Naturgy)} & \multicolumn{1}{c|}{Enresa}      & PWR    \\ \hline
    \rowcolor[HTML]{ECF4FF} 
    \multicolumn{1}{|c|}{\cellcolor[HTML]{ECF4FF}\textbf{Potencia térmica}} &
      \textbf{Potencia eléctrica} &
      \multicolumn{2}{c|}{\cellcolor[HTML]{ECF4FF}\textbf{Refrigeración}} &
      \multicolumn{2}{c|}{\cellcolor[HTML]{ECF4FF}\textbf{Autorización construcción}} \\ \hline
    \multicolumn{1}{|c|}{510 MWt}         & 160 MWe         & \multicolumn{2}{c|}{Abierta - Río Tajo}                & \multicolumn{2}{c|}{24 de junio de 1964}  \\ \hline
    \rowcolor[HTML]{ECF4FF} 
    \multicolumn{2}{|c|}{\cellcolor[HTML]{ECF4FF}\textbf{Autorización de puesta en marcha}} &
      \multicolumn{2}{c|}{\cellcolor[HTML]{ECF4FF}\textbf{Cese de explotación}} &
      \multicolumn{2}{c|}{\cellcolor[HTML]{ECF4FF}\textbf{Autorización desmantelamiento}} \\ \hline
    \multicolumn{2}{|c|}{11 de octubre de 1968}             & \multicolumn{2}{c|}{30 de abril de 2006}               & \multicolumn{2}{c|}{1 de febrero de 2010} \\ \hline
    \end{tabular}%
    }
    \caption{Características y fechas clave de la Central Nuclear José Cabrera (\cite{csn_info_zorita}).}
    \label{tabla:resumen_zorita}
    \end{table}

\subsubsection{Comparación de la Central Nuclear de Zorita con el reactor AP300} \label{comparacion_zorita_ap300}

Pendiente...

\subsubsection{El Simulador Gráfico Interactivo de Zorita (SGIZ)}

El \acrshort{sgiz} es un simulador gráfico interactivo de alcance total de la Central Nuclear de Zorita. Durante los años de operación de la central, se empleó para el entrenamiento de los operadores de la sala de control, para comprender y analizar la dinámica de la planta y para desarrollar y validar sus procedimientos operativos de emergencia.

En abril de 2008 se firmó el convenio de colaboración entre Unión Fenosa y la UPM para la creación del Aula ``José Cabrera'', donde se instaló el \acrshort{sgiz}. Este aula se ubica en el Departamento de Ingeniería Nuclear de la ETSII-UPM y está dedicada a la enseñanza de la Tecnología de la Operación de las Centrales Nucleares.

El simulador proporciona las respuestas de la planta durante la operación normal, transitorios y condiciones de accidente, basadas en el código TRAC y un conjunto de pantallas ilustrativas en tiempo real, así como un conjunto de alarmas en un panel similar al de la sala de control real de la central nuclear, permitiendo la operación automática y manual en tiempo real de los componentes del sistema completo tanto en condiciones normales como de emergencia.