%%%%%%%%%%%%%%%%%%% - ANEXOS - %%%%%%%%%%%%%%%%%%%

\newpage
\section*{ANEXOS} \label{sec:anexos} % Se añade un asterisco a \section para que el título no esté numerado.
\phantomsection
\addcontentsline{toc}{section}{ANEXOS} % Al utilizar \section* se ha de añadir manualmente el apartado al índice (Table Of Contents, TOC).
\markright{ANEXOS} % Al utilizar \section* se ha de añadir manualmente el título del apartado al encabezado.

\renewcommand{\thesubsection}{\Alph{subsection}} % Se numeran los anexos con letras del alfabeto en lugar de números.
% Se indica que las tablas, figuras y códigos se numeran con el código del anexo (A, B, C, ...) seguido del número de tabla, figura o código dentro del anexo (tabla A.2, figura C.1, etc.)
\renewcommand{\thetable}{\Alph{subsection}.\arabic{table}}
\renewcommand{\thefigure}{\Alph{subsection}.\arabic{figure}}
\renewcommand{\thecode}{\Alph{subsection}.\arabic{code}}

% ---------------- Primer anexo ---------------- %
\setcounter{subsection}{0}
\setcounter{table}{0}
\setcounter{figure}{0}

\subsection{Código} \label{sec:codigo}

Todas las gráficas de las simulaciones realizadas en el \acrshort{sgiz} han sido elaboradas con \textit{Python} a partir de los datos generados en \textit{Excel} por el propio simulador. A continuación, se muestra, a modo de ejemplo, el código para la obtención de una de las gráficas:

\vspace{-5pt}

\begin{code}[H]
\begin{lstlisting}[firstnumber=1, breakindent=55pt]
  # Importación de las librerías necesarias
  import matplotlib.pyplot as plt
  import numpy as np
  import pandas as pd

  # Obtención de datos:
  gen_vapor_camara_imp=pd.read_excel("simulacion3.xlsx","gen_vapor_camara_imp")
  df_gen_vapor_camara_imp=pd.DataFrame=gen_vapor_camara_imp

  tiempo=[]
  for i in range(0,len(df_gen_vapor_camara_imp.tiempo)):
      tiempo.append(df_gen_vapor_camara_imp.tiempo[i].strftime('%H:%M:%S'))
    
  pres_cam_imp=[]
  for i in range(0,len(df_gen_vapor_camara_imp.pres_cam_imp)):
      pres_cam_imp.append(df_gen_vapor_camara_imp.pres_cam_imp[i])

  presion_gen_vapor=[]
  for i in range(0,len(df_gen_vapor_camara_imp.               presion_gen_vapor)):
      presion_gen_vapor.append(df_gen_vapor_camara_imp.presion_gen_vapor[i])

  nivel_gen_vapor=[]
  for i in range(0,len(df_gen_vapor_camara_imp.nivel_gen_vapor)):
      nivel_gen_vapor.append(df_gen_vapor_camara_imp.nivel_gen_vapor[i])

  # Creación del gráfico
  plt.plot(tiempo, presion_camara_impulsos, label='Presión de la cámara de impulsos $(kg/cm^2)$', color='lightcoral')
  plt.plot(tiempo, presion_gen_vapor, label='Presión del generador de vapor $(kg/cm^2)$', color='sandybrown')
  plt.plot(tiempo, nivel_gen_vapor, label='Nivel del generador de vapor R.E. $(cm)$', color='mediumseagreen')

  # Creación de la leyenda y el título
  plt.legend(loc='best')
  plt.xlabel('Tiempo (hh:mm:ss)', family='Times New Roman', size=12)
  plt.title('COMPORTAMIENTO GENERADOR DE VAPOR Y CÁMARA DE IMPULSOS', fontname='Times New Roman', size=18, weight='bold')
  plt.grid(True, color='lightgrey')
  plt.yticks(np.arange(-10,50,5))
  plt.xlim([0, len(tiempo)])
  plt.xticks(np.arange(0,len(tiempo),360))
  plt.xticks(rotation = 10)

  # Mostrar el gráfico
  plt.show()
\end{lstlisting}
\vspace{-5pt}
\caption{Ejemplo del código utilizado para generar las gráficas de las simulaciones. Este en concreto corresponde al código de la figura xx.}
\label{cod:codigo_graficas}
\end{code}