\newpage
\section{CONCLUSIONES Y LÍNEAS FUTURAS} \label{conclusiones}

\subsection{Conclusiones}

Tras finalizar este proyecto, son muchas las conclusiones que pueden extraerse del mismo, comenzando por que se han podido cumplir la totalidad de objetivos inicialmente propuestos. A continuación, se detallan las conclusiones del proyecto en relación a los 3 bloques fundamentales en los que se ha dividido el mismo.

En relación a los \underline{\textbf{\acrlongpl{smr}}}, el análisis realizado sobre el estado del arte ---su historia, sus prestaciones y ventajas, sus aplicaciones y la clasificación de los distintos tipos existentes--- ha permitido un aprendizaje y puesta al día de los aspectos más relevantes de esta tecnología cada vez más empleada y demandada. Las principales conclusiones de este bloque son las siguientes:

\begin{itemize}
    \item Se ha comprobado que, fruto de su compacta configuración y la incorporación de los sistemas de seguridad pasivos más avanzados, \textbf{presentan importantes ventajas en lo relativo a la seguridad} tanto en condiciones de operación normal como en transitorios indeseados.
    \item Frente a la mayor inversión inicial necesaria y al gran reto tecnológico que supone un diseño adecuado de los distintos módulos, se ha demostrado que \textbf{la modularización de estos reactores pequeños aumenta la maniobrabilidad y productividad a la hora de fabricar sus componentes y construirlos}.
    \item El análisis de los combustibles avanzados empleados, ha permitido el aprendizaje de diversas formas de inducir las reacciones de fisión en cadena necesarias para el funcionamiento de la planta, permitiendo algunas de ellas el reciclado del combustible ``gastado''.
    \item Junto con las ventajas relativas al tiempo de construcción fruto del diseño integral y modular de los mismos, existen importantes ventajas económicas. Con el estudio económico realizado ---en el que se han comparado los costes de estos pequeños reactores con los de las grandes centrales nucleares convencionales-- se concluye lo siguiente: \textbf{pese a que los costes de la primera unidad construida de un tipo de \acrshort{smr} siempre son más elevados, la modularidad, el diseño simplificado y la estandarización en la fabricación de los mismos resulta en una importante reducción del coste por unidad de energía producida}.
    \item A todo esto se suma su \textbf{gran versatilidad para ser empleados en aplicaciones tan necesarias como el seguimiento de carga, la cogeneración, el abastecimiento eléctrico de zonas remotas e, incluso, la exploración espacial}. La primera aplicación mencionada ---el seguimiento de carga---, es la que se ha analizado con mayor profundidad, tal y como se proponía en los objetivos del proyecto. Sin embargo, se ha podido comprobar que el campo de aplicación del resto de utilidades mencionadas es igualmente muy amplio y de gran interés actual.
    \item En relación al anterior punto, es tan grande el interés que múltiples países muestran con respecto a esta tecnología que existen en todo el mundo más de 80 diseños de \acrshortpl{smr} comerciales en distintas fases de desarrollo. La clasificación de todos estos diseños en sus tipologías correspondientes ha facilitado la comprensión del amplio abanico de posibilidades a la hora de fabricar un \acrshort{smr}.
    \item Por último, todas estas propiedades analizadas que se han mencionado hasta ahora, se han puesto de manifiesto a la hora de estudiar un \acrlong{smr} que se encuentra en un avanzado grado de desarrollo: el AP300 de Westinghouse. La investigación sobre este reactor concreto ha facilitado la comprensión en profundidad del funcionamiento y modo de implementación de los sistemas de seguridad pasivos más recientes, junto con el conocimiento de la regulación, los tiempos y los costes involucrados en la consecución real de este tipo de proyectos.
\end{itemize}

Con respecto a los \underline{\textbf{simuladores}}, se extraen las siguientes conclusiones de la clasificación y estudio realizado:

\begin{itemize}
    \item Existen todo tipo de simuladores focalizados en objetivos, usuarios, tipos de tecnología, alcance y niveles de complejidad concretos. Dentro de la clasificación realizada, \textbf{el \acrshort{sgiz} se encuentra dentro de los simuladores de alcance total de tipo gráfico interactivo para la formación profesional}.
    \item Concretamente, en el campo de los \acrshort{smr}, se están desarrollando múltiples simuladores para el aprendizaje del manejo y el funcionamiento de los mismos, incluyendo \acrshortpl{smr} multiunidad. De toda la oferta disponible, el breve estudio de los simuladores de este tipo ofrecidos por NuScale y por Tecnatom demuestra que son herramientas idóneas para la formación en este campo.
    \item La investigación llevada a cabo sobre el uso de simuladores en la enseñanza ha permitido obtener una amplia lista de ventajas. Todas ellas giran en torno a una misma realidad: \textbf{el aprendizaje activo y práctico es más efectivo que la enseñanza basada únicamente en clases teóricas, por lo que los simuladores son una herramienta clave para complementar la teoría en la que se fundamentan.}
    \item Finalmente, mediante la profundización sobre las características de la Central Nuclear de José Cabrera ---realizada, en gran medida, gracias a la documentación presente en el \acrshort{sgiz}---, se ha podido comparar esta planta con el \acrshort{smr} anteriormente mencionado: el AP300. En resumen, \textbf{la central nuclear de Zorita presenta muchas características idénticas o muy similares al AP300, mientras que la principal diferencia entre ambas tecnologías es la diferencia generacional de las mismas}. Zorita se encuentra ``desactualizada'' con respecto al AP300 en lo que se refiere al diseño, los materiales y sistemas de seguridad, protección y control avanzados incorporados.
\end{itemize}
  
En lo que a las \underline{\textbf{simulaciones}} se refiere, cabe destacar que han comprendido la parte más interesante y, al mismo tiempo, más desafiante del proyecto. Han permitido el aprendizaje de muchas lecciones tanto teóricas como prácticas, que se resumen a continuación a modo de conclusiones:

\begin{itemize}
    \item Del estudio previo a las simulaciones para el conocimiento teórico de lo que se pretendía simular ---el seguimiento de carga---, se concluye que, pese a que las centrales nucleares de gran escala actuales pueden realizar ese tipo de maniobras, \textbf{los \acrshortpl{smr} tienen capacidades avanzadas de seguimiento de carga que los hace idóneos para operar junto con fuentes de energía renovable intermitente}, favoreciendo así la estabilidad y flexibilidad del sistema eléctrico.
    \item La primera simulación realizada ---100 - 50 - 100\% de potencia en 45 minutos---, a parte de suponer la primera toma de contacto con el simulador tras la formación previa en el mismo, ha permitido el análisis y la comprensión de conceptos esenciales de las centrales nucleares:
    \begin{itemize}
        \item Funcionamiento general de la planta en operación normal.
        \item Modo de variar la potencia de una central nuclear.
        \item Comportamiento \textit{reactor sigue a turbina}.
        \item Efectos del coeficiente de temperatura negativa del moderador.
        \item Inserción de reactividad negativa mediante las barras de control y el ácido bórico. 
        \item Evolución de las variables más importantes en el funcionamiento del generador de vapor y del presionador.
        \item Tasas de variación de potencia en la maniobra realizada, permitiendo la comparación de las mismas con la capacidad de seguimiento de carga de los \acrshortpl{smr}.
    \end{itemize}
    \item A partir de la implementación de esta primera simulación como práctica académica de la asignatura de Tecnologías Avanzadas en Reactores Nucleares, se concluye lo siguiente:
    \begin{itemize}
        \item La satisfacción general de los alumnos con el desarrollo de la práctica ha sido buena.
        \item Para próximas ocasiones, la práctica no debe realizarse en las últimas semanas de curso, por la cantidad de trabajos y exámenes que se acumulan en esa época.
        \item El nivel de dificultad y de focalización en los \acrshort{smr} podría aumentarse para los alumnos de máster y es adecuado para los alumnos de 4º de grado.
        \item Se ha demostrado, a partir del \textit{feedback} aportado por los alumnos, la conclusión previamente comentada en relación con los simuladores: el aprendizaje activo y práctico es más efectivo que la enseñanza basada únicamente en clases teóricas.
    \end{itemize} 
    \item Las simulaciones largas fallidas han servido para aprender varios aspectos técnicos importantes ---incluida alguna limitación--- sobre el simulador y su configuración.
    \item La simulación larga de adaptación a la generación solar (maniobra 100 - 20 - 100\% de potencia en 9 horas y 45 minutos de duración), ha aportado nuevas conclusiones con respecto a las maniobras de seguimiento de carga:
    \begin{itemize}
        \item El sistema de control automático del reactor presenta inestabilidades al trabajar a muy bajas potencias.
        \item El efecto del xenón es significativo ante variaciones de potencia y el sistema automático de control automático del reactor responde adecuadamente, principalmente mediante el empleo de las barras de control y del ácido bórico. Este tipo de simulaciones es, por tanto, muy útil para comprender esta realidad en los reactores nucleares.
    \end{itemize} 
    \item La última simulación larga realizada (la maniobra de adaptación a la curva de demanda eléctrica de 15 horas y 15 minutos de duración), a parte de poner de manifiesto todas las conclusiones mencionadas para las anteriores simulaciones, ha puesto a prueba al simulador y, por tanto, a la Central Nuclear de Zorita, para llevar acabo operaciones rápidas y complejas de seguimiento de carga. El reactor ha respondido adecuadamente y se han alcanzado tasas de variación de potencia muy similares a las de los \acrshortpl{smr} de tipo \acrshort{pwr}, propiciando la siguiente conclusión clave: \textbf{las simulaciones de seguimiento de cargas realizadas en el \acrshort{sgiz} permiten la mejor comprensión de esta aplicación en los \acrlongpl{smr} y una buena aproximación al modo de operación y comportamiento general de los mismos ante este tipo de maniobras}.
    \item De esta manera, \textbf{este Trabajo Fin de Grado, al haber generado una práctica académica aplicable a varias asignaturas de grado y máster, supone el enriquecimiento de estas asignaturas y el aprovechamiento de un recurso potentísimo y últimamente poco utilizado: el \acrshort{sgiz}}.
\end{itemize}

A parte de todas las conclusiones anteriores, cabe destacar que este proyecto ha sido de gran agrado y utilidad para el autor al haber adquirido múltiples conocimientos tanto técnicos como de búsqueda de información, organización, enseñanza, programación y redacción. Asimismo, ha sido una gran satisfacción poder poner múltiples habilidades y conocimientos obtenidos durante la carrera al servicio de un proyecto de investigación concreto sobre un tema de gran importancia actual: los \acrlongpl{smr}.

\subsection{Líneas futuras}

Los trabajos futuros que se proponen al finalizar este proyecto son los siguientes:

\begin{itemize}
    \item Implementación de la práctica de seguimiento de carga en la asignatura de Centrales Nucleares de 4º de grado.
    \item Optimización de la práctica para que involucre activamente a todos los alumnos asistentes a la misma y todos tengan la oportunidad de operar el reactor.
    \item Optimización del desarrollo de la práctica para que tenga mayor focalización en \acrshortpl{smr} para los alumnos de Tecnologías Avanzadas en Reactores Nucleares del máster.
    \item Investigación y realización de otro tipo de simulaciones en el \acrshort{sgiz} que permitan mejorar la comprensión de determinadas ventajas que presentan los \acrlongpl{smr}.
\end{itemize}